\chapter*{ВВЕДЕНИЕ}
\addcontentsline{toc}{chapter}{ВВЕДЕНИЕ}

Последние исследования показывают, что зимой 2022 года россияне летали на горнолыжные курорты на треть чаще. Эксперты оценивали число купленных билетов на горнолыжные курорты и выяснилось, что российские курорты стали популярнее на 32.6\%, а заграничные — на 40\% \cite{stat}.

В связи с этим растет спрос на функциональные и удобные приложения для горнолыжных курортов, которые помогали бы туристам ориентироваться на трассах, планировать свои маршруты по склонам с учетом текущих погодных условий и загруженности подъемников.

Цель курсовой работы -- разработка базы данных для онлайн-мониторинга состояния трасс и подъемников горнолыжного курорта. 



Чтобы достигнуть поставленной цели, требуется решить следующие задачи:

\begin{itemize}
	\item проанализировать существующие решения;
	\item формализовать задачу и данные;
	\item проанализировать способы хранения данных и системы управления базами данных, выбрать наиболее подходящие решения для поставленной задачи;
	\item спроектировать и разработать базу данных;
    \item  реализовать программное обеспечение, которое позволит получить доступ к данным посредством REST API \cite{api};
    \item провести исследование зависимости времени обработки данных от их объема и от распределения вычислений между базой данных и приложением.
\end{itemize}

